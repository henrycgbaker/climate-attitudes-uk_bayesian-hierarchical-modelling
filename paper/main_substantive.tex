\documentclass[12pt]{article}
\usepackage[utf8]{inputenc}
\usepackage[table]{xcolor}
\usepackage{tcolorbox}
\usepackage{hyperref}
\usepackage{graphicx}
\usepackage{float}  
\usepackage[section]{placeins}
\renewcommand{\textfraction}{0.05}
\renewcommand{\floatpagefraction}{0.8}
\graphicspath{{figures/qoi/}{figures/diagnostic/}{figures/}}
\usepackage{amsmath, amssymb, graphicx, booktabs, geometry, setspace, url}
\usepackage[style=apa, backend=biber, natbib=true]{biblatex}
\addbibresource{references.bib}
\geometry{margin=1in}
\setstretch{1.5}

\title{Mapping Multidimensional Climate Attitudes in Britain: A Bayesian Hierarchical Latent Trait Approach}
\author{Henry Baker}
\date{\today}

\begin{document}

\pagenumbering{gobble}

\maketitle

\begin{abstract}
This research note introduces a Bayesian hierarchical latent trait model to analyse multidimensional public attitudes toward climate policy in Britain. Using a nationally representative survey of 3,000 respondents collected in 2025, we estimate three latent dimensions: (1) \textit{Environmentalism}, reflecting prioritisation of environmental protection over economic growth; (2) \textit{Economic Optimism}, capturing individuals' confidence in economic prospects; and (3) \textit{Support for Radical Reform}, indicating preferences for abrupt systemic governance and policy changes over maintaining the status quo. Our hierarchical approach integrates individual demographic characteristics and variation across political parties and regions. Preliminary results highlight some notable demographic effect patterns — particularly around material insecurity and age (plus a lack of educational effects), alongside significant party-level differences. Additionally, we observe a residual correlation between environmentalism and economic optimism, which holds implications for targeted climate communication strategies emphasising optimistic narratives. Methodologically, we illustrate how hierarchical Bayesian latent trait modelling provides a coherent framework for quantifying complex opinion structures and assessing their predictors. Further model refinement, sensitivity checks, and downstream population-averaged predictive-analysis are needed to evaluate the robustness of the initial findings herein, and to support practical model application and audience segmentation.
\end{abstract}

\begin{center}
\textbf{GitHub repo}: \url{https://github.com/henrybaker/climate-attitudes-uk_bayesian-hierarchical-modelling}    
\end{center}

\newpage


\tableofcontents

\vfill

\begin{tcolorbox}
    \textbf{Note:} Throughout we do not report 95\% credible intervals alongside our posterior point estimates, rather we present full posterior distributions in associated plots. This approach reflects a core strength of Bayesian modelling: the coherent propagation of uncertainty throughout the model. Rather than emphasising arbitrary thresholds for statistical significance (e.g.\ $p < 0.05$), our visualisation of the entire posterior density aims to convey a more intuitive and informative picture of uncertainty around each parameter estimate.
\end{tcolorbox}

\newpage

\pagenumbering{arabic}

\section{Introduction}

Accurate measurement of these complex latent attitudes is crucial for understanding opinion polarisation and designing effective climate policy and communication strategies. Noisy single-item survey measures inadequately capture the nuanced nature of environmental orientations, and scholars have long advocated multi-item scales and latent trait models, such as the New Ecological Paradigm (NEP), to reliably measure underlying environmental worldviews (Dunlap et al., 2000; Treier \& Jackman, 2008). Moreover, hierarchical modelling approaches (SEM) further facilitate simultaneous estimation of individual-level and group-level influences on attitudes (Gelman \& Hill, 2007; Fox, 2010), and techniques such as multilevel regression and post-stratification (MRP) have previously been used to map and explain regional variations in climate attitudes (Howe et al., 2015).

\section{Data \& Measures}

We analyse data from a nationally representative UK survey conducted in 2025. From the survey items, we identified three clusters of questions corresponding to Environmentalism (prioritised over growth) ($\theta$), Economic Optimism ($\phi$), and support for Radical Reform ($\psi$). Despite the ordinal nature of responses, we treat each as a continuous indicator of the latent factor after z-score standardisation, though a cumulative probit/logit model would be more appropriate. 

All items were coded with higher values indicating more of the latent trait, which simplifies interpretation and allows us to constrain factor‐loading signs for identifiability.  We assess unidimensionality by computing the posterior mean communality (\(\mathrm{comm}_j\)) and uniqueness (\(\mathrm{uniq}_j\)) for each item (where $\mathrm{comm}_j + \mathrm{uniq}_j = \mathrm{Var}(Y_j)$) (Table~\ref{tab:comm_uniq}). The percentage of variance explained for each block is comfortably over 60\% (at 76\%, 77\%, and 79\% for respectively), suggesting the unidimensional factor structure provides an adequate summary of the item responses within each construct. Similarly, looking at the individual factor loadings, there was some heterogeneity, especially in the radical-reform block where some candidates for removal were apparent (see Figure ~\ref{fig:item_discrim}, but overall all items performed satisfactorily.

Based on prior research on climate attitudes and political behaviour, we include a set of covariates to explain variation in the latent traits across individuals. These include gender, age, and education (all dummy encoded), as well as self-reported perceptions of material insecurity (where we constructed an index from eight questions asking how frequently the respondent experienced certain financial stresses). We treat party affiliation and region as grouping factors with random intercepts. 

\section{Model Specification}

The model consists of two parts: three measurement models that link the observed survey responses to the latent traits, and a structural model that specifies how the latent traits vary by individual covariates and group-level effects. The entire model is estimated jointly in a Bayesian framework, allowing us to propagate uncertainty from the measurement part into our estimates of substantive effects.

For each latent trait ($\phi,\theta,\psi$), we assume a linear \emph{item response model}. Given the continuous/standardised nature of our recoded item responses, we use a Gaussian factor analytic formulation (see Section~\ref{sec:limitations}). The combined measurement model can be written as:

\begin{align*}
(\text{Optimism: }\phi)\quad &Y^{(\phi)}_{i j}
  = \alpha^{(\phi)}_j + \lambda^{(\phi)}_j\,\phi_i + \varepsilon^{(\phi)}_{i j}, 
  &&\varepsilon^{(\phi)}_{i j}\sim\mathcal{N}\bigl(0,\;\sigma^{2\,(\phi)}_j\bigr),
  &j=1,\dots,6,\\
(\text{Environmentalism: }\theta)\quad &Y^{(\theta)}_{i k}
  = \alpha^{(\theta)}_k + \lambda^{(\theta)}_k\,\theta_i + \varepsilon^{(\theta)}_{i k}, 
  &&\varepsilon^{(\theta)}_{i k}\sim\mathcal{N}\bigl(0,\;\sigma^{2\,(\theta)}_k\bigr),
  &k=1,\dots,5,\\
(\text{Radicalism: }\psi)\quad &Y^{(\psi)}_{i \ell}
  = \alpha^{(\psi)}_\ell + \lambda^{(\psi)}_\ell\,\psi_i + \varepsilon^{(\psi)}_{i \ell}, 
  &&\varepsilon^{(\psi)}_{i \ell}\sim\mathcal{N}\bigl(0,\;\sigma^{2\,(\psi)}_\ell\bigr),
  &\ell=1,\dots,8.
\end{align*}

For identifiability, we restrict each $\lambda \geq 0$, and impose a scale constraint by fixing unit variance for each latent trait, ensuring $\Omega$ is a correlation matrix (i.e. diagonal elements = 1). Posterior diagnostics confirm the MCMC algorithm did not suffer from sign-switching or scale drifting (see Figure~\ref{fig:trace}).

We model the structural component as a multivariate hierarchical regression, giving the following dimensions, where $(\xi_{i,1}, \xi_{i,2}, \xi_{i,3})^\top \sim \mathcal{N}(\mathbf{0},\Omega)$ are the individual-level residuals:

\begin{equation*}
\boldsymbol{\eta}_i \;\sim\; \mathcal{N}_3\Bigl(\underbrace{\boldsymbol{\alpha}_{r_i} + \boldsymbol{\delta}_{q_i}}_{\text{Region + Party intercepts}} \;+\; B\,\mathbf{X}_i,\;\; \Omega \Bigr)
\label{eq:structural_mvn}
\end{equation*}

\begin{align*}
\phi_i &= \alpha_{r_i,1} + \delta_{q_i,1} + B_{1,\cdot}\,\mathbf{X}_i + \xi_{i,1},\\
\theta_i &= \alpha_{r_i,2} + \delta_{q_i,2} + B_{2,\cdot}\,\mathbf{X}_i + \xi_{i,2},\\
\psi_i &= \alpha_{r_i,3} + \delta_{q_i,3} + B_{3,\cdot}\,\mathbf{X}_i + \xi_{i,3},
\end{align*}

We use a non-centered parameterisation for the latent trait vector $\boldsymbol{\eta}_i$ to improve sampling efficiency and convergence; specifically, we sample raw standard-normal draws $\mathbf{z}_i \sim \mathcal{N}(\mathbf{0}, I_3)$ and transform them as $\boldsymbol{\eta}_i = \mu_i + L_{\eta} \mathbf{z}_i$, where $\mu_i = \boldsymbol{\alpha}_{r_i} + \boldsymbol{\delta}_{q_i} + B \mathbf{X}_i$ is the mean structure and $L_{\eta}$ is the Cholesky factor of the residual correlation matrix $\Omega$.

\subsection{Priors \& Estimation}

For each item $j$ in the optimism, environment, and radical-reform blocks, we place weakly informative priors $\beta_j^{(\cdot)}\sim\mathcal{N}(0,0.5)$ on factor loadings, $\lambda_j^{(\cdot)}\sim\mathrm{Lognormal}(\log 1,0.2)$ (with $\lambda_j>0$) on item-level scales, and $\sigma_j^{(\cdot)}\sim\mathcal{N}(1,0.2)$ (with $\sigma_j>0$) on residual variances. Each covariate effect (i.e., each slope in the matrix $B$) is likewise given $\beta^{(\ell)}_{p}\sim\mathcal{N}(0,0.5)$, concentrating most standardized effects within approximately $\pm1$. For group-level intercepts, we adopt a non-centered parameterisation: for each region $r=1,\dots,R$, draw $\boldsymbol{\alpha}^{\mathrm{raw}}_r\sim\mathcal{N}_3(\mathbf{0},I_3)$, set $\sigma_{\alpha,\ell}\sim\mathcal{N}^{+}(0,0.1)$ and $L_{\alpha}\sim\mathrm{LKJ}(2)$, and then form $\boldsymbol{\alpha}_r = D_\alpha\,L_\alpha\,\boldsymbol{\alpha}^{\mathrm{raw}}_r$ with $D_\alpha=\mathrm{diag}(\sigma_{\alpha,1},\sigma_{\alpha,2},\sigma_{\alpha,3})$. An identical construction applies for party intercepts $\boldsymbol{\delta}_q$ over $q=1,\dots,Q$, using raw draws $\boldsymbol{\delta}^{\mathrm{raw}}_q\sim\mathcal{N}_3(\mathbf{0},I_3)$, $\sigma_{\delta,\ell}\sim\mathcal{N}^{+}(0,0.1)$, and $L_{\delta}\sim\mathrm{LKJ}(2)$, with $\boldsymbol{\delta}_q = D_\delta\,L_\delta\,\boldsymbol{\delta}^{\mathrm{raw}}_q$ and $D_\delta=\mathrm{diag}(\sigma_{\delta,1},\sigma_{\delta,2},\sigma_{\delta,3})$. Finally, latent residual variation $\eta_i$ is parameterised via marginal standard deviations $\tau_{\eta,\ell}$ (with $\tau_{\eta,1},\tau_{\eta,2}\sim\mathcal{N}^{+}(0,0.3)$ and $\tau_{\eta,3}\sim\mathcal{N}^{+}(0,0.1)$) and a Cholesky factor $L_{\eta}\sim\mathrm{LKJ}(2)$; the full residual covariance is then $\Omega = L_\eta\,\mathrm{diag}(\tau_{\eta})\,\mathrm{diag}(\tau_{\eta})^\top\,L_\eta^\top$. 

These priors were iteratively refined via prior predictive checks, and although further tuning (and more appropriate likelihood specifications/assumptions) could additionally improve the priors’ coverage of the observed data in the prior predictive stage, our posterior predictive checks confirm that the model successfully recovers the key group‐level patterns of interest and replicates observed data distributions well. Remaining discrepancies appear primarily at the item-response level rather than the group level, reflecting the Gaussian measurement assumption’s limitations in matching the exact shape of individual item distributions - however these do not compromise our higher-level substantive inferences. Prior- and posterior-predictive checks are visualised in Appendix~\ref{app:diagnostics}). The model is implemented in Stan where we run four parallel MCMC chains using a NUTS sampler for 3000 iterations each, discarding the first 1000 as warm‐up. Convergence diagnostics indicate no inference problems, with $\hat{R} < 1.01$ for all parameters (Figure ~\ref{fig:r-hat}, and both bulk and tail effective sample sizes (ESS) $\geq$ 200 for all investigated parameters. Similarly, no chains hit the max tree depth, nor were there problematically divergent traditions, all chains exhibited sufficient energy exploration with $\mathrm{BFMI} > 0.2$, and trace plots indicate good mixing. Prior predictive checks confirmed the priors to be reasonable, and posterior predictive checks show that the model replicates observed data distributions well.

\section{Results}

Overall reliabilities for each factor are high (see Table~\ref{tab:reliabilities}; \(\Omega\) reliability for each factor $\geq$ 0.7), again underscoring that item batteries coherently capture their intended constructs and that a one-factor model fit each item set well enough for our purposes. The three latent dimensions remain moderately interrelated even after accounting for individual-level covariates and group-level effects (Table~\ref{tab:resid_corr} for residual correlation point estimates; Figure~\ref{fig:corr_dist} for full posterior distributions).\footnote{\textbf{Note}: we do not report the \textit{marginal} correlation between latent traits, although in retrospect this might have provide a more intuitive descriptive quantity.} Conditional on age, education, and party, the correlation between optimism and environmentalism is moderately positive ($\rho_{\phi,\theta} \approx +0.37$), whereas both optimism and radicalism, and environmentalism and radicalism, are negatively associated ($\rho_{\phi,\psi} \approx -0.48$; $\rho_{\theta,\psi} \approx -0.28$), suggesting that support for radical systemic change is not a defining feature of most environmentalists in the sample.

Party affiliation explains substantial between-party variation in all three latent dimensions (see Table~\ref{tab:party_intercepts} for party-group random intercept posterior mean point estimates for each latent trait, and Figure~\ref{fig:party_intercept} for distributions). For environmentalism ($\theta$), non-Conservative parties exceed the global mean, yet it is striking that the Greens do not lead the field (behind LibDems and Labour). This could reflect a covariate‐selection artefact - if Greens skew along one of the traits already in the model, then that trait's direct coefficient might soak up some of their environmental advantage. Re-estimating the model without key demographics or computing and comparing population-average predictions by party would be necessary to unpack further. Radicalism is highest among fringe and non-mainstream supporters, and economic optimism ($\phi$) is strongest for Labour and Conservative voters (i.e. mainstream voters). The wider uncertainty for parties like the SNP and Plaid Cymru likely reflects smaller sample sizes and greater within-party heterogeneity for nationally-associated parties. 

Our region intercepts (Table~\ref{tab:region_intercepts} and Figure~\ref{fig:region_intercept}) reveal only modest geographic variation across all three latent dimensions once demographics and party affiliation are accounted for. Environmentalism ($\theta$) intercepts range from –0.067 (Wales) to +0.076 (London), a span of 0.14 standard deviations - small compared to party‐group differences - as was the span of 0.16 standard deviations for Economic Optimism ($\phi$) intercepts, while all of the Radical‐Reform ($\psi$) intercepts' are effectively zero once other predictors are included (i.e. their 95\% credible intervals fail to exclude the null). After accounting for demographics and party, region contributes little additional variation to respondents’ latent attitudes.



Individual-level covariate effects (Table~\ref{tab:covariate_effects} \& Figure~\ref{fig:corr_dist}) reveal that, contrary to expectation (once party and region intercept effects have been account for) economic insecurity aligns with greater environmental concern rather than lower, and that older respondents - while less optimistic and less environmentalist - are actually more open to radical reform. Education's effect is ambiguous across classes and not statistically significant except at the university degree level. Notably, party affiliation accounts for more variance in all three traits than any single demographic predictor. Future extensions could calculate marginal versus conditional $R^2$ to formally decompose the relative contributions of fixed (demographic) and random (party/region) effects. In the full model, optimism shows the highest explained variance, with over 60\% of variation in economic outlook accounted for by our predictors - reflecting strong alignment of economic attitudes with demographic and group‐level factors. Environmental concern is moderately well explained, while radical‐reform attitudes remain the most idiosyncratic (with only $\approx$ 22\% explained). If drivers of populism and anti-status quo attitudes are of substantive interest, our structural model should be expanded.

\section{Discussion \& Policy Implications}

Climate concern is broad, political, and nuanced; contrary to conventional wisdom, environmentalism (\(\theta\)) is highest among Liberal Democrats (+0.17) and Labour supporters (+0.14), with Greens in third place (+0.09). This pattern suggests pro-environment concern has diffused across the main centre-left parties. Further diagnostics are needed to investigate the drivers of this effect, which may look different under robustness/sensitivity checks (i.e. omitting certain demographics) or population-averaged predictions. Accordingly, communicators should frame climate action in ways that resonates with centre/centre-right values. Nevertheless, the Greens underperformance on the environment remains striking. As above, one interpretation is that climate concern is no longer distinctive to the Greens; another is that some of those who vote Green are motivated by a broader anti-establishment ethos (captured by $\psi$) and are perhaps disillusioned former Labour voters, whose primary driver isn’t always environmentalism \textit{per se}. Indeed, whereas Labour's intercept on radicalism was noticeably down at -0.26, the Green's was positive but not extreme, and was less radical than the “Other” non-partisans and Reform UK. This is consistent with the mechanism whereby the Greens attract a mix of dedicated environmentalists and more general protest voters. 

Demographically, older cohorts score lower on environmentalism (65+ at –0.19) but, higher on radical reform (65+ at +0.26), meanwhile, a university degree confers only a small boost to environmentalism (+0.06; not statistically significant). Messaging around substantive reform could resonate strongly with older demographics, while younger cohorts - already optimistic and pro-environment - may require more advanced, participatory engagement. Similarly, orthogonal to the the conventional belief that material security is a pre-requisite for supporting potentially economically-disruptive climate policy, our model finds that higher material insecurity is in fact associated with \emph{greater} environmental concern (+0.16) and slightly \emph{higher} optimism (+0.05), with no statistically significant effect on radicalism. Accordingly, framing green policies around economic opportunity - such as job creation and cost savings - could effectively mobilise these insecure but optimistically pro-climate constituencies. 

The multi-dimensional nature of public attitudes captured by $\phi$, $\theta$, $\psi$ allow climate communicators to segment their audience and tailor messages accordingly. The positive correlation between environmentalism and optimism (\(\rho_{\phi,\theta}\approx+0.375\)) and the negative link to radicalism (\(\rho_{\theta,\psi}\approx-0.278\)) imply that \emph{optimistic environmentalists} (high \(\theta\), high \(\phi\), low \(\psi\)) form a large, reachable segment. By contrast, \emph{radical pessimists} (high \(\psi\), low \(\phi\), low \(\theta\)) are relatively rare and concentrated among fringe voters. Broad climate campaigns should emphasise hopeful, practical solutions, while niche channels may be needed to engage those favouring systemic overhaul. Our model could be further leveraged for targeted messaging and audience segmentation by further computing downstream population‐averaged predictions - both conditional on given covariates and via post‐stratification (MRP) - or by generating latent‐score predictions for hypothetical respondents defined by specific covariate profiles (see Appendix~\ref{app:profiles} for two examples). 

\section{Limitations and Future Directions}
\label{sec:limitations}

Our approach made several simplifying assumptions. First, we treated ordinal survey items as continuous, applying a linear factor model. While convenient and commonly done, a more rigorous approach would use an ordinal IRT using a cumulative probit/logistic model (e.g., a graded response model with threshold parameters for Likert items). We judged this acceptable given our focus on group differences (which tend to be robust to such transformations) and to keep the model manageable. We anticipate the substantive conclusions would remain, however an ordinal latent trait model would give improved fit and perhaps slightly different estimates for extreme respondents. Second, we assumed the latent traits are normally distributed (by the multivariate normal in the structural model). In reality, public opinion distributions can be skewed or multimodal (perhaps a large mass of moderately pro-environment people and a small cluster of very sceptical people). Our model might not capture heavy skewness perfectly. Posterior predictive checks did indicate some slight underestimation of extreme responses on a couple of items, which could hint at latent skewness. A more flexible approach could use non-parametric mixing or allow heavier-tailed distributions. 

Furthermore, at a modelling level, our current model includes only additive effects, potentially averaging out important heterogeneity. Future extensions could incorporate interactions, such as covariate-by-covariate effects (e.g., economic insecurity and education jointly affecting environmentalism), covariate-by-group interactions (e.g., by allowing the group intercepts and individual slopes themselves to have correlations - such as differing age effects across political parties), or even latent-by-latent interactions (e.g., the combined effect of environmentalism and radical reform attitudes on policy support).

At a metrics level, we report an overall Bayesian $R^2$ for each trait under the full model (Table~\ref{tab:r^2}), but it would have been useful to distinguish between Marginal $R^2$ and Conditional $R^2$. A full variance decomposition would tell us, for example, what share of variance in environmentalism is due exclusively to demographics, to party affiliation, to region, or to their overlap. This would have useful/actionable implications for targeting of strategic communications and audience segmentation. 

Finally, all data used herein are self-reported attitudes at one point in time. This means that the cross-sectional data does not benefit from dynamic or causal modelling; additionally, social desirability bias could affect responses on environmental questions. If such bias varies by group (e.g., maybe younger, educated respondents feel more pressure to say the socially desirable pro-climate answer), it could inflate estimated differences. 

\subsection{Future Work}

In addition to implementing ordinal IRT, interaction effects, and dynamic modelling, several further extensions could enhance the analytical scope and interpretability of our model. First, we could leverage demographic and party-composition data through Multilevel Regression and Poststratification (MRP), estimating detailed distributions of latent attitudes within each parliamentary constituency. Additionally, adopting a multivariate response extension would allow us to model responses to concrete climate-policy support questions (e.g., endorsement of carbon taxes or renewable energy subsidies) jointly rather than individually. Formally, this involves specifying a multivariate response model such as

$$
\mathbf{y}_i \sim \mathrm{MVN}\left(\boldsymbol{\mu}_i,\, \Sigma_y\right), \quad\text{where}\quad \boldsymbol{\mu}_i = \Lambda\,\boldsymbol{\eta}_i + B\,\mathbf{x}_i.
$$

Here, $\mathbf{y}_i = (y_{i,1},\dots,y_{i,K})$ represents an individual's responses to a set of $K$ climate-policy support items, and $\boldsymbol{\eta}_i = (\phi_i,\theta_i,\psi_i)$ denotes the vector of latent traits (optimism, environmentalism, radicalism). This multivariate approach enables us to disentangle precisely how each latent dimension influences specific policy preferences — for instance, determining whether environmentalism ($\theta$) or radicalism ($\psi$) more strongly predicts support for ambitious initiatives such as a Green New Deal. Extending the model further to incorporate actual policy outcomes or observed behavioural data would bridge the gap between latent attitudes and concrete actions (attitude-behaviour congurence). For example, we could model local adoption of climate policies as outcomes driven by the regional distributions of latent traits, formally represented as hierarchical regression:

$$
\text{policy outcome}_{r} \sim f(\boldsymbol{\alpha}_r, \text{regional covariates}),
$$

where $\boldsymbol{\alpha}_r$ denotes region-level latent trait intercepts. Translating latent trait measures into interpretable metrics in this way would facilitate communication.

\section{Conclusion}

This research note demonstrates the utility of Bayesian hierarchical latent trait models for capturing nuanced public attitudes toward climate policy in Britain. From a methodological perspective, our hierarchical Bayesian approach effectively captures individual-level variation and group-level heterogeneity across political parties and regions, and by partial pooling of group-level intercepts, the model shrinks extreme estimates toward the overall mean, improving stability for small-sample categories and avoiding overfitting with this relatively complex model. Our findings offer valuable practical guidance for policymakers and climate communicators aiming to build broad, resilient coalitions for environmental action by embracing multi-dimensional audience segmentation and framing strategies.

\section{Included Plots}

\begin{figure}[ht]
    \centering    
  \includegraphics[width=0.9\linewidth]{party_group_effects_distributions}
  \caption{Party‐group intercepts (posteriors distributions)}
    \label{fig:party_intercept}
\end{figure}

\begin{figure}[ht]
    \centering    
  \includegraphics[width=0.9\linewidth]{region_group_effects_distributions}
  \caption{Party‐group intercepts (posteriors distributions)}
    \label{fig:party_intercept}
\end{figure}

\begin{figure}[ht]
    \centering    
  \includegraphics[width=0.85\linewidth]{covariate_effects_overlapping_ridgelines.png}
    \label{fig:corr_dist}
\end{figure}

\newpage
\appendix
\section{Appendixes}

\subsection{Model Diagnostics}
\label{app:diagnostics}

\subsubsection{Prior Predictive Checks}
\label{sec:prior_predic}

\begin{figure}[H]
    \centering
    \includegraphics[width=0.75\linewidth]{prior_predictive_environment_density.png}
    \label{fig:enter-label}
    \caption{}
\end{figure}

\begin{figure}[H]
    \centering
    \includegraphics[width=0.75\linewidth]{prior_predictive_environment_person_means.png}
    \label{fig:enter-label}
\end{figure}

\begin{figure}[H]
    \centering
    \includegraphics[width=0.75\linewidth]{prior_predictive_optimism_density.png}
    \label{fig:enter-label}
\end{figure}

\begin{figure}[H]
    \centering
    \includegraphics[width=0.75\linewidth]{prior_predictive_optimism_person_means.png}
    \label{fig:enter-label}
\end{figure}

\begin{figure}[H]
    \centering
    \includegraphics[width=0.75\linewidth]{prior_predictive_radical_density.png}
    \label{fig:enter-label}
\end{figure}

\begin{figure}[H]
    \centering
    \includegraphics[width=0.75\linewidth]{prior_predictive_radical_person_means.png}
    \label{fig:enter-label}
\end{figure}

\subsubsection{Posterior Predictive Checks: Category Frequencies}

\begin{figure}[H]
    \centering
    \includegraphics[width=0.75\linewidth]{category_frequencies_optimism_item1.png}
    \label{fig:enter-label}
\end{figure}

\begin{figure}[H]
    \centering
    \includegraphics[width=0.75\linewidth]{category_frequencies_radical_item1.png}
    \label{fig:enter-label}
\end{figure}


\subsubsection{Posterior Predictive Checks}

%env

\begin{figure}[H]
    \centering
    \includegraphics[width=0.75\linewidth]{posterior_predictive_environment_density.png}
    \label{fig:enter-label}
\end{figure}

\begin{figure}[H]
    \centering
    \includegraphics[width=0.75\linewidth]{item_level_means_environment.png}
    \label{fig:enter-label}
\end{figure}

\begin{figure}[H]
    \centering
    \includegraphics[width=0.75\linewidth]{posterior_predictive_environment_person_means.png}
    \label{fig:enter-label}
\end{figure}

% opt

\begin{figure}[H]
    \centering
    \includegraphics[width=0.75\linewidth]{posterior_predictive_optimism_density.png}
    \label{fig:enter-label}
\end{figure}

\begin{figure}[H]
    \centering
    \includegraphics[width=0.75\linewidth]{item_level_means_optimism.png}
    \label{fig:enter-label}
\end{figure}

\begin{figure}[H]
    \centering
    \includegraphics[width=0.75\linewidth]{posterior_predictive_optimism_person_means.png}
    \label{fig:enter-label}
\end{figure}

% rad

\begin{figure}[H]
    \centering
    \includegraphics[width=0.75\linewidth]{posterior_predictive_radical_density.png}
    \label{fig:enter-label}
\end{figure}

\begin{figure}[H]
    \centering
    \includegraphics[width=0.75\linewidth]{item_level_radicalreform.png}
    \label{fig:enter-label}
\end{figure}

\begin{figure}[H]
    \centering
    \includegraphics[width=0.75\linewidth]{posterior_predictive_radical_person_means.png}
    \label{fig:enter-label}
\end{figure}

\subsubsection{Posterior Checks: Distributional Shape}

\begin{figure}[H]
    \centering
    \includegraphics[width=0.65\linewidth]{densities_selected_params.png}
    \label{fig:enter-label}
    \caption{Each density is estimated from the full posterior and illustrates both central tendency and tail behaviour; 95\% credible intervals (not shown) confirm adequate posterior concentration under our weakly informative priors}
\end{figure}

\subsubsection{MCMC Convergence and Trace Plots}

\begin{figure}[H]
    \centering
    \includegraphics[width=1\linewidth]{rhat_histogram.png}
    \label{fig:r-hat}
    \caption{Histogram of R-hat values for all parameters - no values above 1.01 threshold}
\end{figure}


\begin{figure}[H]
    \centering
    \includegraphics[width=0.75\linewidth]{traceplots_selected_params.png}
    \label{fig:trace}
    \caption{Trace plots for selected parameters (\(\sigma_{\delta,1}\), \(\rho_{\phi,\theta}\), and a representative \(\beta\)). Chains mix well and show no divergences.}
\end{figure}

\subsection{Item Communality, Uniquenesses}
\label{app:communality_unique}

\begin{table}[H]
  \centering
  \begin{tabular}{llrr}
    \toprule
    Block            & Item & $\mathrm{Communality}$ & $\mathrm{Uniqueness}$ \\
    \midrule
    Environment      & 1    & $0.75$  & $0.51$ \\
    Environment      & 2    & $2.1$   & $0.15$ \\
    Environment      & 3    & $0.098$ & $0.91$ \\
    Environment      & 4    & $1.8$   & $0.20$ \\
    Environment      & 5    & $1.7$   & $0.23$ \\
    Optimism         & 1    & $1.3$   & $0.15$ \\
    Optimism         & 2    & $1.1$   & $0.25$ \\
    Optimism         & 3    & $0.86$  & $0.36$ \\
    Optimism         & 4    & $1.1$   & $0.21$ \\
    Optimism         & 5    & $0.48$  & $0.60$ \\
    Optimism         & 6    & $1.2$   & $0.19$ \\
    Radical-Reform   & 1    & $2.6$   & $0.21$ \\
    Radical-Reform   & 2    & $1.3$   & $0.40$ \\
    Radical-Reform   & 3    & $0.55$  & $0.63$ \\
    Radical-Reform   & 4    & $3.1$   & $0.16$ \\
    Radical-Reform   & 5    & $2.2$   & $0.25$ \\
    Radical-Reform   & 6    & $3.5$   & $0.14$ \\
    Radical-Reform   & 7    & $0.064$ & $0.94$ \\
    Radical-Reform   & 8    & $0.30$  & $0.76$ \\
    \bottomrule
  \end{tabular}
    \caption{Communalities and uniquenesses by block and item (posterior means)}
    \label{tab:comm_uniq}
\end{table}

\begin{table}[H]
  \centering
  \begin{tabular}{lccc}
    \toprule
    Block             & Mean Communality & Mean Uniqueness & \% Variance Explained \\
    \midrule
    Environment       & 1.29             & 0.40             & 76\% \\
    Optimism          & 0.99             & 0.29             & 77\% \\
    Radical‐Reform    & 1.69             & 0.44             & 79\% \\
    \bottomrule
  \end{tabular}
\caption{Average communality, uniqueness, and explained variance by item block}
  \label{tab:comm_uniq_summary}
\end{table}

\begin{figure}[H]
    \centering
    \includegraphics[width=1\linewidth]{item_discrimination_lambda_distirbutions.png}
    \caption{Item discriminations by trait; there's variation in informativeness and some of the lower $\lambda$ values suggest possible redundancy / candidates for revision or removal}
    \label{fig:item_discrim}
\end{figure}


\subsection{Factor Reliability}
\label{app:reliability}

\begin{table}[H]
    \centering
    \begin{tabular}{ll}
    \toprule
    Item block & Mean $\omega$ \\
    \midrule
    $\omega_{env}$ & 0.90 \\
    $\omega_{opt}$ & 0.94 \\
    $\omega_{rad}$ & 0.93 \\
    \bottomrule
    \end{tabular}
    \caption{Factor Reliabilities}
    \label{tab:reliabilities}
\end{table}

\subsection{Residual Latent Correlations}
\label{app:correlations}

\begin{table}[H]
  \centering
  \begin{tabular}{lc}
    \toprule
    Pair & Correlation \\
    \midrule
    $\phi$--$\theta$ &  0.37 \\
    $\phi$--$\psi$   & -0.48 \\
    $\theta$--$\psi$ & -0.28 \\
    \bottomrule
  \end{tabular}
  \caption{Latent Trait Residual Correlations}
  \label{tab:resid_corr}
\end{table}

\begin{figure}[H]
    \centering
    \includegraphics[width=0.75\linewidth]{latent_corr_with_point_estimate.png}
    \label{fig:corr_dist}
\end{figure}

\subsection{Party‐Intercepts}
\label{app:party}

\begin{table}[H]
  \centering
  \label{tab:party_intercepts}
  \begin{tabular}{lccc}
    \toprule
    Party            & $\boldsymbol{\phi}$ Optimism & $\boldsymbol{\theta}$ Environmentalism & $\boldsymbol{\psi}$ Radical-Reform \\
    \midrule
    Green            & $-0.092$   & $+0.092$   & $+0.027$   \\
    Labour           & $+0.51$    & $+0.14$    & $-0.26$    \\
    Plaid Cymru      & $-0.11$    & $+0.0071$  & $+0.066$   \\
    SNP              & $-0.16$    & $-0.068$   & $+0.076$   \\
    Liberal Democrat & $+0.21$    & $+0.17$    & $-0.13$    \\
    Conservative     & $+0.26$    & $+0.030$   & $-0.0097$  \\
    Reform UK        & $-0.23$    & $-0.22$    & $+0.13$    \\
    Another party    & $-0.28$    & $-0.044$   & $+0.074$   \\
    Don’t know       & $+0.075$   & $+0.030$   & $+0.010$   \\
    Won’t vote       & $+0.010$   & $-0.064$   & $-0.073$   \\
    \bottomrule
  \end{tabular}
  \caption{Party-group intercepts for latent traits (posterior means)}
\end{table}

\begin{figure}[H]
    \centering    
  \includegraphics[width=1.1\linewidth]{party_group_effects_distributions}
  \caption{Party‐group intercepts (posteriors distributions)}
    \label{fig:party_intercept}
\end{figure}

\subsection{Region‐Intercepts}
\label{app:region}

\begin{table}[ht]
  \centering
  \begin{tabular}{lccc}
    \toprule
    Region                    & $\boldsymbol{\phi}$ Optimism & $\boldsymbol{\theta}$ Environmentalism & $\boldsymbol{\psi}$ Radical-Reform \\
    \midrule
    Yorkshire \& the Humber   & $-0.053$  & $+0.0016$  & $-0.0026$   \\
    West Midlands             & $-0.033$  & $-0.0057$  & $-0.0025$   \\
    Scotland                  & $-0.0030$ & $-0.036$   & $-0.0056$   \\
    Wales                     & $-0.021$  & $-0.067$   & $-0.0053$   \\
    North West                & $+0.0055$ & $+0.013$   & $+0.0039$   \\
    Eastern                   & $+0.020$  & $+0.047$   & $+0.0038$   \\
    South West                & $-0.039$  & $-0.0021$  & $+0.00060$  \\
    East Midlands             & $+0.036$  & $-0.0050$  & $+0.0052$   \\
    London                    & $+0.11$   & $+0.076$   & $+0.0014$   \\
    South East                & $-0.016$  & $-0.017$   & $+0.00020$  \\
    \bottomrule
  \end{tabular}
    \caption{Region-group intercepts on latent traits (posterior means)}
    \label{tab:region_intercepts}
\end{table}

\begin{figure}[H]
    \centering    
  \includegraphics[width=1.1\linewidth]{region_group_effects_distributions}
  \caption{Region‐group intercepts (posterior distributions)}
    \label{fig:region_intercept}
\end{figure}

\subsection{Individual-level Demographic Covariate Effects}

\begin{table}[ht]
  \centering
  \begin{tabular}{lccc}
    \toprule
    Covariate                          & $\boldsymbol{\phi}$ Optimism & $\boldsymbol{\theta}$ Environmentalism & $\boldsymbol{\psi}$ Radical-Reform \\
    \midrule
    \textbf{Gender}                    &                              &                                        &                                     \\
    \quad Male (reference)                   & $0.000$                      & $0.000$                                & $0.000$                             \\
    \quad Female                       & $-0.148$                     & $-0.022$                               & $+0.037$                            \\
    \addlinespace
    \textbf{Age}                       &                              &                                        &                                     \\
    \quad 18–24 (reference)                  & $0.000$                      & $0.000$                                & $0.000$                             \\
    \quad 25–34                        & $+0.068$                     & $-0.003$                               & $+0.061$                            \\
    \quad 35–44                        & $-0.004$                     & $-0.030$                               & $+0.113$                            \\
    \quad 45–54                        & $-0.251$                     & $-0.181$                               & $+0.202$                            \\
    \quad 55–64                        & $-0.314$                     & $-0.190$                               & $+0.216$                            \\
    \quad 65+                          & $-0.377$                     & $-0.194$                               & $+0.261$                            \\
    \addlinespace
    \textbf{Education}                &                              &                                        &                                     \\
    \quad No qualifications (reference)     & $0.000$                      & $0.000$                                & $0.000$                             \\
    \quad Apprenticeship               & $-0.130$                     & $-0.110$                               & $+0.056$                            \\
    \quad Other                        & $-0.100$                     & $-0.034$                               & $+0.008$                            \\
    \quad Level 1                      & $-0.100$                     & $+0.053$                               & $-0.015$                            \\
    \quad Level 2                      & $-0.102$                     & $-0.007$                               & $-0.017$                            \\
    \quad Level 3                      & $-0.166$                     & $-0.049$                               & $+0.015$                            \\
    \quad Level 4+                     & $+0.082$                     & $+0.057$                               & $-0.059$                            \\
    \addlinespace
    \textbf{Material insecurity (1 SD $\uparrow$)} & $+0.048$      & $+0.155$                               & $+0.006$                            \\
    \bottomrule
  \end{tabular}
  \vspace{1ex}
  \caption{Posterior mean covariate effects on latent traits.}
  \label{tab:covariate_effects}
  \footnotesize
  \textbf{Note:} Values are posterior mean coefficients; individual covariates are standardised or coded as 0/1 dummy variables, so their coefficients can be interpreted in roughly comparable units (the education and age effects are differences relative to a baseline category).
\end{table}

\begin{figure}[H]
    \centering    
  \includegraphics[width=0.85\linewidth]{covariate_effects_overlapping_ridgelines}
  \caption{Demographic-covariate effects Heatmap}
    \label{fig:covar_effect_heatmap}
\end{figure}

\begin{figure}[H]
    \centering    
  \includegraphics[width=\linewidth]{covariate_effects_heatmap}
  \caption{Demographic-covariate effect Posteriors Distributions}
    \label{fig:covar_effects}
\end{figure}

\subsection{Variance Explained (R\(^2\))}
\label{app:variance}

\begin{table}[ht]
  \centering
    \begin{tabular}{ll}
    \toprule
    block & mean $R^2$ \\
    \midrule
    $R^2_{env}$ & 0.39 \\
    $R^2_{opt}$ & 0.61 \\
    $R^2_{rad}$ & 0.22 \\
    \bottomrule
    \end{tabular}
    \caption{Point estimates of Bayesian $R^2$ for each latent dimension, under the full model}
    \label{tab:r^2}
\end{table}

\begin{figure}[H]
    \centering
    \includegraphics[width=0.95\linewidth]{r2_posteriors.png}
    \caption{Posterior $R^2$ Distributions}
    \caption{Full distributions of Bayesian $R^2$ for each latent dimension, under the full model}
    \label{fig:r^2_dist}
\end{figure}


\subsection{Hypothetical Personas}
\label{app:personas}

\begin{enumerate}
    \item A 65-year-old male Reform UK supporter with no qualifications, one-SD above the mean in material insecurity, living in the South West (Figure~\ref{fig:person_profile_1}). His predicted latent (point-estimate) scores are  
    \[
    \hat\phi \;\approx\;-0.60,\quad
    \hat\theta \;\approx\;-0.26,\quad
    \hat\psi \;\approx\;+0.40.
    \]
    \begin{figure}[H]
        \centering
        \includegraphics[width=0.75\linewidth]{example_target_profiles/Elderly_Male_Reform_UK_No_Qualifications_High_Insecurity_South_East.png}
        \label{fig:person_profile_1}
    \end{figure}

    \item  A 45–54-year-old female Labour supporter with a university degree, one-SD below the mean in material insecurity, living in the North West. (Figure~\ref{fig:person_profile_2}) Her predicted (point-estimate) scores are  
    \[
    \hat\phi \;\approx\;+0.15,\quad
    \hat\theta \;\approx\;-0.15,\quad
    \hat\psi \;\approx\;-0.08.
    \]

    \begin{figure}[H]
        \centering
        \includegraphics[width=0.75\linewidth]{example_target_profiles/Mid_Age_Female_Labour_Univ_Grad_Low_Insecurity_North_West.png}
        \label{fig:person_profile_2}
    \end{figure}
\end{enumerate}

\subsection{Regional \& Party Profiles}
\label{app:profiles}

\begin{figure}[H]
    \centering
    \includegraphics[width=0.75\linewidth]{region_psi_vs_phi_col_env.png}
    \label{fig:region_profile_1}
\end{figure}

\begin{figure}[H]
    \centering
    \includegraphics[width=0.75\linewidth]{region_psi_vs_theta_col_opt.png}
    \label{fig:region_profile_2}
\end{figure}

\begin{figure}[H]
    \centering
    \includegraphics[width=0.75\linewidth]{region_theta_vs_phi_col_rad.png}
    \label{fig:region_profile_3}
\end{figure}

\begin{figure}[H]
  \centering
  \includegraphics[width=\textwidth]{party_radar_charts.png}yes
  \caption{Party profiles as radar charts in attitude \((\phi,\theta,\psi)\) space.}
  \label{fig:party_radar}
\end{figure}





\newpage

\nocite{*}
\printbibliography

\end{document}
