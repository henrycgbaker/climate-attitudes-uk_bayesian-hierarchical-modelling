\documentclass[a4paper,11pt]{article}
\usepackage[top=2.5cm,bottom=2.5cm,left=2.5cm,right=2.5cm]{geometry}
\usepackage[T1]{fontenc}
\usepackage[utf8]{inputenc}
\usepackage{multirow,booktabs,graphicx,enumerate,float,fancyhdr,amsmath,amssymb,tcolorbox,xcolor,array,makecell}
\setlength{\parindent}{0in}
\setlength{\parskip}{1em}
\pagestyle{fancy}
\fancyhf{}
\lhead{\footnotesize Bayesian Modelling – Research Note}
\rhead{\footnotesize Henry Baker}
\cfoot{\footnotesize \thepage}
\begin{document}

\thispagestyle{empty}
\begin{tabular}{p{15.5cm}}
{\large \bf Bayesian Modelling\\\hline}
\end{tabular}

\begin{center}
{\Large \bf Research Note}\\
1 - Model Specification\\
Henry Baker (228755)\\
\end{center}

\section{Data and Preprocessing}

\subsection{Survey Items and Raw Coding}

\begin{enumerate}
  \item \textbf{Economic‐Optimism Items (6 Likert questions)}  
    \begin{itemize}
      \item Questions (e.g.\ “To what extent do you feel optimistic or pessimistic about \(\dots\)?”) have response categories 
        \[
          \{\text{Very pessimistic},\;\text{Fairly pessimistic},\;\text{Neither},\;\text{Fairly optimistic},\;\text{Very optimistic}\}.
        \]
      \item Recode each item \(j\) for respondent \(i\):
        \[
          \text{opt}_{ij} 
          =
          \begin{cases}
            -2, & \text{Very pessimistic},\\
            -1, & \text{Fairly pessimistic},\\
             0, & \text{Neither},\\
             1, & \text{Fairly optimistic},\\
             2, & \text{Very optimistic},
          \end{cases}
          \quad j=1,\dots,6.
        \]
      \item For each item \(j\), compute its sample mean \(\overline{\text{opt}}_{\!\cdot j}\) and SD \(\mathrm{SD}(\text{opt}_{\cdot j})\), then standardize:
        \[
          Y^{(\text{opt})}_{ij}
          \;=\;
          \frac{\text{opt}_{ij} \;-\;\overline{\text{opt}}_{\!\cdot j}}%
               {\mathrm{SD}(\text{opt}_{\cdot j})},
          \quad i=1,\dots,N,\;j=1,\dots,6.
        \]
      \item Denote \(\mathbf{Y}^{(\text{opt})} = [Y^{(\text{opt})}_{ij}]\in\mathbb{R}^{N\times 6}\).
    \end{itemize}

  \item \textbf{Environmental‐Priority Items (1 forced‐choice + 4 Likert)}  
    \begin{itemize}
      \item \emph{Forced‐choice item (Q121)}:  
        \[
          \begin{aligned}
            &\text{Statement A: “Cost‐first”}\quad\bigl(\text{‘‘Action to reduce the cost of living has to come first over sustainability and being eco‐friendly.’’}\bigr),\\
            &\text{Statement B: “Environment‐first”}\quad\bigl(\text{‘‘Protecting the natural environment is more important than rebuilding infrastructure.’’}\bigr).
          \end{aligned}
        \]
        Recode
        \[
          \text{env}_{i0} =
          \begin{cases}
            0, & \text{if respondent chooses cost‐first (Statement A)},\\
            1, & \text{if respondent chooses environment‐first (Statement B)}.
          \end{cases}
        \]
        Then standardize: 
        \(\widetilde{\text{env}}_{i0} = \bigl(\text{env}_{i0} - \overline{\text{env}}_{\!\cdot 0}\bigr)/\mathrm{SD}(\text{env}_{\cdot 0})\).
      \item \emph{Four Likert items (Q122–Q125)}:  
        \[
          \{\text{Strongly disagree},\;\text{Somewhat disagree},\;\text{Neither},\;\text{Somewhat agree},\;\text{Strongly agree}\}.
        \]
        One of these (the cost‐first framing) is reverse‐coded.  Let \(k^\ast\) index that item.  For each \(k=1,\dots,4\):
        \[
          \text{env}_{ik}^{(\text{raw})} =
          \begin{cases}
            -2, & \text{Strongly disagree},\\
            -1, & \text{Somewhat disagree},\\
             0, & \text{Neither},\\
             1, & \text{Somewhat agree},\\
             2, & \text{Strongly agree}.
          \end{cases}
        \]
        For the item \(k^\ast\) phrased in the “cost‐first” direction, multiply \(\text{env}_{i,k^\ast}^{(\text{raw})}\) by \(-1\) so that higher values always indicate greater pro‐environment priority.  Then standardize each column:
        \[
          \widetilde{\text{env}}_{ik} = \frac{\text{env}_{ik}^{(\text{raw})} - \overline{\text{env}}_{\!\cdot k}}{\mathrm{SD}(\text{env}_{\cdot k})}, 
          \quad k=1,\dots,4.
        \]
      \item Stack the forced‐choice and the four Likert columns to form 
        \(\mathbf{Y}^{(\text{env})} = [\,\widetilde{Y}^{(\text{env})}_{ik}\,]\in\mathbb{R}^{N\times 5},\;k=0,\dots,4.\)
    \end{itemize}

  \item \textbf{Radical‐Reform Items (8 forced‐choice questions)}  
    For each respondent \(i\), there are eight pairs of statements (Q72–Q75; Q78–Q79; Q82–Q83).  We code each as follows: if respondent chooses the statement expressing a radical‐reform orientation, set \(\psi^{(\text{raw})}_{i\ell} = 1\); if chooses the status‐quo statement, set \(\psi^{(\text{raw})}_{i\ell} = 0\).  Concretely, index \(\ell=1,\dots,8\) for:

    \[
      \begin{aligned}
        &\ell=1:\;\text{(Q72) “Westminster moved too slowly…” (radical) vs “…moved at a good pace…” (status quo)},\\
        &\ell=2:\;\text{(Q73) “Looking for politicians who get things done” (radical) vs “…take time for long‐term goals” (status quo)},\\
        &\ell=3:\;\text{(Q74) “Support those prepared to take radical action…” (radical) vs “…keep system broadly the same” (status quo)},\\
        &\ell=4:\;\text{(Q75) “Britain is broken and needs radical action” (radical) vs “…is broadly doing fine” (status quo)},\\
        &\ell=5:\;\text{(Q78) “UK not taking right steps…needs change” (radical) vs “…taking right steps” (status quo)},\\
        &\ell=6:\;\text{(Q79) “UK headed wrong and needs radical reform” (radical) vs “…headed in right direction” (status quo)},\\
        &\ell=7:\;\text{(Q82) “Acceptable for unelected bodies to decide” (radical) vs “Decisions should lie with elected reps” (status quo)},\\
        &\ell=8:\;\text{(Q83) “Politicians have allowed unelected bodies too much power” (radical) vs“…Trust experts/unelected officials” (status quo).}
      \end{aligned}
    \]
    Then for each \(\ell\), compute sample mean \(\overline{\psi}^{(\text{raw})}_{\!\cdot\ell}\) and SD \(\mathrm{SD}(\psi^{(\text{raw})}_{\cdot\ell})\), and standardize:
    \[
      Y^{(\text{rad})}_{i\ell}
      = 
      \frac{\psi^{(\text{raw})}_{i\ell} - \overline{\psi}^{(\text{raw})}_{\!\cdot\ell}}%
           {\mathrm{SD}(\psi^{(\text{raw})}_{\cdot\ell})},
      \quad \ell=1,\dots,8.
    \]
    Denote \(\mathbf{Y}^{(\text{rad})} = [\,Y^{(\text{rad})}_{i\ell}\,]\in\mathbb{R}^{N\times 8}.\)

  \item \textbf{Material‐Insecurity Items (8 frequency items; Q90–Q97)}  
    \begin{itemize}
      \item Each of eight items (e.g.\ “I worry about meeting my bills”) has response categories 
        \(\{\text{Never},\;\text{Occasionally},\;\text{Fairly often},\;\text{Very often}\}.\)
      \item Recode 
        \[
          \text{insec}^{(\text{raw})}_{i\ell} =
          \begin{cases}
            0, & \text{Never},\\
            1, & \text{Occasionally},\\
            2, & \text{Fairly often},\\
            3, & \text{Very often},
          \end{cases}
          \quad \ell=1,\dots,8.
        \]
      \item Compute composite 
        \[
          \text{insec}^\ast_i = \frac{1}{8}\sum_{\ell=1}^8 \text{insec}^{(\text{raw})}_{i\ell}.
        \]
      \item Standardize: 
        \(\displaystyle \text{insec}_i = \frac{\text{insec}^\ast_i - \overline{\text{insec}}^\ast}{\mathrm{SD}(\text{insec}^\ast)}.\)
        Thus \(\text{insec}_i\in\mathbb{R}\) is a continuous covariate.
    \end{itemize}

  \item \textbf{Demographic Covariates}  
    \begin{itemize}
      \item \emph{Gender:}  
        \[
          \text{gen}_i =
          \begin{cases}
            0, & \text{Male},\\
            1, & \text{Female}.
          \end{cases}
        \]
      \item \emph{Age Bracket (6 levels)}: Original codes \(1,\dots,6\) correspond to 
        \(\{65+,\,55\text{–}64,\,45\text{–}54,\,35\text{–}44,\,25\text{–}34,\,18\text{–}24\}.\)  
        Choose “18–24” (code 6) as reference.  Define five dummies:
        \[
          \begin{aligned}
            \text{age}_{i,65+} &= 1_{\{\text{orig}_i=1\}},\quad
            \text{age}_{i,55-64} = 1_{\{\text{orig}_i=2\}},\\
            \text{age}_{i,45-54} &= 1_{\{\text{orig}_i=3\}},\quad
            \text{age}_{i,35-44} = 1_{\{\text{orig}_i=4\}},\quad
            \text{age}_{i,25-34} = 1_{\{\text{orig}_i=5\}}.
          \end{aligned}
        \]
        If “18–24,” then all age dummies = 0.
      \item \emph{Education Level (7 categories)}: Original codes \(1,\dots,7\) denote 
        \(\{\text{Level 4+},\,\text{Level 2},\,\text{Level 1/Entry},\,\text{Level 3},\,\text{Apprenticeship},\,\text{No qualifications},\,\text{Other}\}.\)
        Choose “No qualifications” (code 6) as reference.  Create six indicators:
        \[
          \begin{aligned}
            \text{edu}_{i,L4+} &= 1_{\{\text{orig}_i=1\}},\quad
            \text{edu}_{i,L2} = 1_{\{\text{orig}_i=2\}},\quad
            \text{edu}_{i,L1} = 1_{\{\text{orig}_i=3\}},\\
            \text{edu}_{i,L3} &= 1_{\{\text{orig}_i=4\}},\quad
            \text{edu}_{i,\text{appr}} = 1_{\{\text{orig}_i=5\}},\quad
            \text{edu}_{i,\text{other}} = 1_{\{\text{orig}_i=7\}}.
          \end{aligned}
        \]
        If original = 6 (“No qualifications”), then all six = 0.
      \item \emph{Party Affiliation (possibly merged categories)}:  
        Let \(\text{vote\_tomorrow}_i,\text{vote\_recent}_i\) be reported.  Define
        \[
          \text{vote}_i = \text{coalesce}(
            \text{vote\_tomorrow}_i,\,
            \text{vote\_recent}_i
          ),
        \]
        then map each party name (e.g.\ Conservative, Labour, Greens, Liberal Democrats, SNP, etc.) –and optionally “Another party,” “Don’t know,” “Won’t vote” (either separate or merged into “Other”)–to an integer \(q_i\in\{1,\dots,Q\}.\)
        This \(q_i\) will index party‐level random intercepts.
      \item \emph{Region (10 levels)}:  
        Map each region string to an integer
        \[
          r_i \in \{1,\dots,10\}
        \]
        according to:
        \[
          1:\text{Yorkshire \& the Humber},\,
          2:\text{West Midlands},\,
          3:\text{Scotland},\,
          4:\text{Wales},\,
          5:\text{North West},\,
          6:\text{Eastern},\,
          7:\text{South West},\,
          8:\text{East Midlands},\,
          9:\text{London},\,
          10:\text{South East}.
        \]
        Use \(r_i\) to index region‐level random intercepts.
    \end{itemize}
\end{enumerate}

\subsection{Final Design Matrices}

\begin{itemize}
  \item \(\mathbf{Y}^{(\text{opt})} \in \mathbb{R}^{N\times 6}\) of standardized economic‐optimism items.
  \item \(\mathbf{Y}^{(\text{env})} \in \mathbb{R}^{N\times 5}\) of standardized environmental‐priority items.
  \item \(\mathbf{Y}^{(\text{rad})} \in \mathbb{R}^{N\times 8}\) of standardized radical‐reform items.
  \item \(\mathbf{X} = [\,\text{gen},\,\text{age}_{65+},\,\text{age}_{55-64},\,\text{age}_{45-54},\,\text{age}_{35-44},\,\text{age}_{25-34},\,\text{edu}_{L4+},\,\text{edu}_{L3},\,\text{edu}_{L2},\,\text{edu}_{L1},\,\text{edu}_{\text{appr}},\,\text{edu}_{\text{other}},\,\text{insec}\,] \in \mathbb{R}^{N\times 13}.\)
  \item Party index vector \(\boldsymbol{q} = (q_1,\dots,q_N)\) with \(q_i\in\{1,\dots,Q\}.\)
  \item Region index vector \(\boldsymbol{r} = (r_1,\dots,r_N)\) with \(r_i\in\{1,\dots,10\}.\)
\end{itemize}

\section{Model Specification}

\subsection{Overview}

We propose a hierarchical Bayesian model with:
\begin{enumerate}
  \item A \textbf{measurement model} for economic‐optimism (\(\phi_i\)) via 6 continuous indicators.
  \item A \textbf{measurement model} for environmental‐priority (\(\theta_i\)) via 5 continuous indicators.
  \item A \textbf{measurement model} for radical‐reform orientation (\(\psi_i\)) via 8 continuous indicators.
  \item A \textbf{structural model} for the joint latent vector 
    \(\boldsymbol{\eta}_i = (\phi_i,\,\theta_i,\,\psi_i)^\top\in\mathbb{R}^3,\)
    regressing on region‐ and party‐level random intercepts and individual‐level covariates \(\mathbf{X}_i\).
\end{enumerate}

\subsection{Notation and Dimensions}

\begin{itemize}
  \item \(N\): total number of respondents.
  \item \(P_{\text{opt}} = 6\): number of optimism items.
  \item \(P_{\text{env}} = 5\): number of environmental items.
  \item \(P_{\text{rad}} = 8\): number of radical‐reform items.
  \item \(R = 10\): number of UK regions.
  \item \(Q\): number of parties.
  \item \(P_{\text{cov}} = 13\): number of individual‐level covariates (gender, age dummies, education dummies, insecurity).
  \item \(\mathbf{Y}^{(\text{opt})} \in \mathbb{R}^{N\times 6}\) with entries \(Y^{(\text{opt})}_{ij}\).
  \item \(\mathbf{Y}^{(\text{env})} \in \mathbb{R}^{N\times 5}\) with entries \(\widetilde{Y}^{(\text{env})}_{ik}\).
  \item \(\mathbf{Y}^{(\text{rad})} \in \mathbb{R}^{N\times 8}\) with entries \(Y^{(\text{rad})}_{i\ell}\).
  \item \(\mathbf{X} \in \mathbb{R}^{N\times 13}\) with rows \(\mathbf{X}_i^\top = (\,\text{gen}_i,\ldots,\text{insec}_i\,).\)
  \item Region index \(\boldsymbol{r} \in \{1,\dots,10\}^N\).
  \item Party index \(\boldsymbol{q} \in \{1,\dots,Q\}^N\).
  \item Latent factors \(\phi_i,\theta_i,\psi_i\), collected into \(\boldsymbol{\eta}_i = (\phi_i,\,\theta_i,\,\psi_i)^\top \in \mathbb{R}^3.\)
\end{itemize}

\subsection{Measurement Models}

\subsubsection{Economic‐Optimism}

For each respondent \(i\) and item \(j=1,\dots,6\),
\begin{equation}
  Y^{(\text{opt})}_{ij} 
  \;\sim\; 
  \mathcal{N}\Bigl(\,\alpha^{(\text{opt})}_j + \lambda^{(\text{opt})}_j\,\phi_i,\;\sigma^{(\text{opt})\,2}_j\Bigr),
  \label{eq:meas_opt}
\end{equation}
where \(\alpha^{(\text{opt})}_j\in\mathbb{R}\), \(\lambda^{(\text{opt})}_j>0\), and \(\sigma^{(\text{opt})}_j>0\).  

\paragraph{Priors:} For \(j=1,\dots,6\),
\[
  \alpha^{(\text{opt})}_j \sim \mathcal{N}(0,1), 
  \quad
  \lambda^{(\text{opt})}_j \sim \mathrm{LogNormal}(0,1), 
  \quad
  \sigma^{(\text{opt})}_j \sim \text{Student‐}t_3(0,1)\quad(\text{half-}t).
\]

\subsubsection{Environmental‐Priority}

For each \(i\) and item \(k=1,\dots,5\),
\begin{equation}
  \widetilde{Y}^{(\text{env})}_{ik} 
  \;\sim\; 
  \mathcal{N}\Bigl(\,\alpha^{(\text{env})}_k + \lambda^{(\text{env})}_k\,\theta_i,\;\sigma^{(\text{env})\,2}_k\Bigr),
  \label{eq:meas_env}
\end{equation}
where \(\alpha^{(\text{env})}_k\in\mathbb{R}\), \(\lambda^{(\text{env})}_k>0\), and \(\sigma^{(\text{env})}_k>0\).

\paragraph{Priors:} For \(k=1,\dots,5\),
\[
  \alpha^{(\text{env})}_k \sim \mathcal{N}(0,1), 
  \quad
  \lambda^{(\text{env})}_k \sim \mathrm{LogNormal}(0,1), 
  \quad
  \sigma^{(\text{env})}_k \sim \text{Student‐}t_3(0,1)\quad(\text{half-}t).
\]

\subsubsection{Radical‐Reform}

For each \(i\) and item \(\ell=1,\dots,8\),
\begin{equation}
  Y^{(\text{rad})}_{i\ell}
  \;\sim\;
  \mathcal{N}\Bigl(\,\alpha^{(\text{rad})}_\ell + \lambda^{(\text{rad})}_\ell\,\psi_i,\;\sigma^{(\text{rad})\,2}_\ell\Bigr),
  \label{eq:meas_rad}
\end{equation}
where \(\alpha^{(\text{rad})}_\ell\in\mathbb{R}\), \(\lambda^{(\text{rad})}_\ell>0\), and \(\sigma^{(\text{rad})}_\ell>0\).

\paragraph{Priors:} For \(\ell=1,\dots,8\),
\[
  \alpha^{(\text{rad})}_\ell \sim \mathcal{N}(0,1), 
  \quad
  \lambda^{(\text{rad})}_\ell \sim \mathrm{LogNormal}(0,1), 
  \quad
  \sigma^{(\text{rad})}_\ell \sim \text{Student‐}t_3(0,1)\quad(\text{half-}t).
\]

\subsection{Structural Model: Trivariate Latent Regression}

Collect \(\phi_i,\theta_i,\psi_i\) into 
\[
  \boldsymbol{\eta}_i 
  = 
  \begin{pmatrix}
    \phi_i \\[4pt]
    \theta_i \\[4pt]
    \psi_i
  \end{pmatrix}
  \;\in\;\mathbb{R}^3.
\]
We assume
\begin{equation}
  \boldsymbol{\eta}_i \;\sim\; \mathcal{N}_3\Bigl(\,
    \underbrace{
      \begin{pmatrix}
        \alpha_{\,r_i,\,1} \\[4pt]
        \alpha_{\,r_i,\,2} \\[4pt]
        \alpha_{\,r_i,\,3}
      \end{pmatrix}
      \;+\;
      \begin{pmatrix}
        \delta_{\,q_i,\,1} \\[4pt]
        \delta_{\,q_i,\,2} \\[4pt]
        \delta_{\,q_i,\,3}
      \end{pmatrix}
    }_{\substack{\text{Region and}\\\text{Party Intercepts}}}
    \;+\;
    \underbrace{B\,\mathbf{X}_i}_{\substack{\text{Covariate}\\\text{Effects}}},
    \;\Omega
  \Bigr),
  \label{eq:structural_mvn}
\end{equation}
where:
\begin{itemize}
  \item \(\alpha_{\,r,\,\ell}\) (\(\ell=1,2,3\)) are region‐level intercepts for latent dimensions \(\phi,\theta,\psi\).
  \item \(\delta_{\,q,\,\ell}\) (\(\ell=1,2,3\)) are party‐level intercepts.
  \item \(\mathbf{X}_i \in \mathbb{R}^{13}\) is the individual‐level covariate vector.
  \item \(B \in \mathbb{R}^{3\times 13}\) has rows \(B_{1,\cdot}\) (effects on \(\phi\)), \(B_{2,\cdot}\) (on \(\theta\)), \(B_{3,\cdot}\) (on \(\psi\)).
  \item \(\Omega \in \mathbb{R}^{3\times 3}\) is the residual covariance matrix, constrained to be a correlation matrix (unit variances).
\end{itemize}

\subsubsection{Non‐Centered Parameterizations}

\paragraph{Latent residuals:} Let \(\eta^{\text{raw}}_i \sim \mathcal{N}_3(\mathbf{0},\,I_3)\).  Then
\[
  \boldsymbol{\eta}_i
  = L_{\eta}\,\eta^{\text{raw}}_i,
  \quad
  L_{\eta}L_{\eta}^\top = \Omega,
  \quad
  L_{\eta} \sim \text{LKJ\_Cholesky}(3).
\]

\paragraph{Region intercepts:} For each region \(r=1,\dots,10\),
\[
  \alpha^{\text{raw}}_r \sim \mathcal{N}_3(\mathbf{0},\,I_3), 
  \quad
  \boldsymbol{\alpha}_r 
  = D_{\alpha}\,L_{\alpha}\,\alpha^{\text{raw}}_r,
\]
where
\[
  D_{\alpha} = \text{diag}\bigl(\sigma_{\alpha,1},\,\sigma_{\alpha,2},\,\sigma_{\alpha,3}\bigr),
  \quad
  \sigma_{\alpha,\ell} \sim \text{Cauchy}_{+}(0,\,2.5),
\]
and \(L_{\alpha}\sim\text{LKJ\_Cholesky}(3)\).  Hence \(\boldsymbol{\alpha}_r \sim \mathcal{N}_3(\mathbf{0},\,\Sigma_{\alpha})\) with \(\Sigma_{\alpha} = D_{\alpha} L_{\alpha}L_{\alpha}^\top D_{\alpha}.\)

\paragraph{Party intercepts:} For each party \(q=1,\dots,Q\),
\[
  \delta^{\text{raw}}_q \sim \mathcal{N}_3(\mathbf{0},\,I_3), 
  \quad
  \boldsymbol{\delta}_q 
  = D_{\delta}\,L_{\delta}\,\delta^{\text{raw}}_q,
\]
where
\[
  D_{\delta} = \text{diag}\bigl(\sigma_{\delta,1},\,\sigma_{\delta,2},\,\sigma_{\delta,3}\bigr),
  \quad
  \sigma_{\delta,\ell} \sim \text{Cauchy}_{+}(0,\,2.5),
  \quad
  L_{\delta} \sim \text{LKJ\_Cholesky}(3).
\]
Thus \(\boldsymbol{\delta}_q\sim\mathcal{N}_3(\mathbf{0},\,\Sigma_{\delta})\).

\subsubsection{Priors on Covariate Slopes}

Vectorize \(B\in\mathbb{R}^{3\times 13}\).  For each latent index \(\ell=1,2,3\) and covariate index \(p=1,\dots,13\),
\[
  \beta^{(\ell)}_{p} 
  \;\sim\; \mathcal{N}(0,\,1).
\]
Hence row 1 of \(B\) contains \(\{\beta^{(\phi)}_p\}\), row 2 contains \(\{\beta^{(\theta)}_p\}\), row 3 contains \(\{\beta^{(\psi)}_p\}\).

\subsubsection{Residual Covariance Prior}

\[
  L_{\eta} \sim \text{LKJ\_Cholesky}(3), 
  \quad
  \eta^{\text{raw}}_i \sim \mathcal{N}_3(\mathbf{0},\,I_3),
  \quad i=1,\dots,N.
\]
Consequently, \(\Omega = L_{\eta}L_{\eta}^\top\) is a correlation matrix with three unit‐variance latent dimensions.

\subsection{Identification Constraints}

\begin{itemize}
  \item \textbf{Scale:} We fix \(\mathrm{Var}(\phi_i) = \mathrm{Var}(\theta_i) = \mathrm{Var}(\psi_i) = 1\) by modeling \(\eta^{\text{raw}}_i \sim \mathcal{N}_3(0,I)\) and setting \(\boldsymbol{\eta}_i = L_{\eta}\,\eta^{\text{raw}}_i\).
  \item \textbf{Location:} Each \(\alpha_r\) and \(\delta_q\) has prior mean \(\mathbf{0}\).  Covariates in \(\mathbf{X}\) are mean‐zero (continuous columns standardized; dummies with reference = 0).  Thus \(E[\phi_i]=E[\theta_i]=E[\psi_i]=0\).
  \item \textbf{Sign:} All loadings \(\{\lambda^{(\text{opt})}_j,\lambda^{(\text{env})}_k,\lambda^{(\text{rad})}_\ell\}\) are strictly positive (via \(\mathrm{LogNormal}(0,1)\)), eliminating sign‐flips.
  \item \textbf{Sufficiency of Indicators:} Each latent has at least 5 indicators (6 for \(\phi\), 5 for \(\theta\), 8 for \(\psi\)), ensuring a well‐posed covariance.
  \item \textbf{Non‐Centered Parametrizations:} For \(\alpha_r\), \(\delta_q\), and \(\eta_i\), we use non‐centered forms to avoid funnel‐geometry issues.
\end{itemize}

\section{Parameter Interpretations}

\subsection{Measurement Parameters}

\begin{itemize}
  \item \(\alpha^{(\text{opt})}_j\): Expected standardized score on optimism‐item \(j\) when \(\phi_i=0\).
  \item \(\lambda^{(\text{opt})}_j\): Strength with which item \(j\) loads on \(\phi_i\).  Larger \(\lambda^{(\text{opt})}_j\) means item \(j\) is a stronger indicator of economic‐optimism.
  \item \(\sigma^{(\text{opt})}_j\): Residual SD of optimism‐item \(j\) given \(\phi_i\).
  \item \(\alpha^{(\text{env})}_k\), \(\lambda^{(\text{env})}_k\), \(\sigma^{(\text{env})}_k\): Analogous interpretations for environmental‐priority items.
  \item \(\alpha^{(\text{rad})}_\ell\): Expected standardized score on radical‐reform item \(\ell\) when \(\psi_i=0\).
  \item \(\lambda^{(\text{rad})}_\ell\): Loading of item \(\ell\) on \(\psi_i\).  Larger values indicate more informative radical‐reform indicators.
  \item \(\sigma^{(\text{rad})}_\ell\): Residual SD of radical‐reform item \(\ell\) given \(\psi_i\).
\end{itemize}

\subsection{Latent Variables}

\begin{itemize}
  \item \(\phi_i\): Respondent \(i\)’s latent economic‐optimism.
  \item \(\theta_i\): Respondent \(i\)’s latent environmental‐priority.
  \item \(\psi_i\): Respondent \(i\)’s latent radical‐reform orientation (higher \(\psi_i\) indicates stronger preference for radical change over status quo).
\end{itemize}

\subsection{Group‐Level Intercepts}

\begin{itemize}
  \item \(\boldsymbol{\alpha}_r = (\alpha_{r,1},\,\alpha_{r,2},\,\alpha_{r,3})^\top\) locates region \(r\) in the 3D latent space.  
    \begin{itemize}
      \item \(\alpha_{r,1}>0 \Rightarrow\) Region \(r\) above average on economic‐optimism.
      \item \(\alpha_{r,2}>0 \Rightarrow\) Region \(r\) above average on environmental‐priority.
      \item \(\alpha_{r,3}>0 \Rightarrow\) Region \(r\) above average on radical‐reform orientation.
    \end{itemize}
  \item \(\boldsymbol{\delta}_q = (\delta_{q,1},\,\delta_{q,2},\,\delta_{q,3})^\top\) locates party \(q\) similarly.
\end{itemize}

\subsection{Covariate Effects}

Let \(B = [\,\beta^{(\ell)}_p\,]_{\ell=1:3,\;p=1:13}\).

\begin{itemize}
  \item \(\beta^{(\phi)}_{p}\) (row 1): Effect of covariate \(p\) on \(\phi_i\).  
  \item \(\beta^{(\theta)}_{p}\) (row 2): Effect of covariate \(p\) on \(\theta_i\).  
  \item \(\beta^{(\psi)}_{p}\) (row 3): Effect of covariate \(p\) on \(\psi_i\).  
  \item In particular, \(\beta^{(\phi)}_{\text{gen}}\) and \(\beta^{(\theta)}_{\text{gen}}\) and \(\beta^{(\psi)}_{\text{gen}}\) measure Female vs Male differences on each latent.  Likewise for each age‐dummy, education dummy, and the continuous material‐insecurity index.
\end{itemize}

\subsection{Residual Covariance}

\(\Omega\) is a \(3\times 3\) correlation matrix:
\[
  \Omega 
  = 
  \begin{pmatrix}
    1 & \rho_{\phi\theta} & \rho_{\phi\psi}\\
    \rho_{\phi\theta} & 1 & \rho_{\theta\psi}\\
    \rho_{\phi\psi} & \rho_{\theta\psi} & 1
  \end{pmatrix}.
\]
\begin{itemize}
  \item \(\rho_{\phi\theta}\): Residual correlation between economic‐optimism and environmental‐priority, conditional on region, party, and covariates.
  \item \(\rho_{\phi\psi}\): Residual correlation between economic‐optimism and radical‐reform orientation.
  \item \(\rho_{\theta\psi}\): Residual correlation between environmental‐priority and radical‐reform orientation.
\end{itemize}

\section{Final Stan Model}

\begin{verbatim}
data {
  int<lower=1> N;                // # of individuals
  int<lower=1> P;                // # of numeric covariates
  matrix[N, P] X;                // standardized covariates

  int<lower=1> R;                           // # of regions
  array[N] int<lower=1, upper=R> region_id; // region index

  int<lower=1> Q;                          // # of parties
  array[N] int<lower=1, upper=Q> party_id; // party index

  int<lower=1> J_opt;                          // # of optimism items
  int<lower=1> N_opt;                          // total optimism responses
  array[N_opt] int<lower=1, upper=N>    i_opt;  // respondent index
  array[N_opt] int<lower=1, upper=J_opt> j_opt; // item index
  array[N_opt] real                     y_opt; // observed (standardized)

  int<lower=1> J_env;                          // # of environment items
  int<lower=1> N_env;                          // total environment responses
  array[N_env] int<lower=1, upper=N>    i_env;  // respondent index
  array[N_env] int<lower=1, upper=J_env> j_env; // item index
  array[N_env] real                     y_env; // observed (standardized)

  int<lower=1> J_rad;                          // # of radical-reform items
  int<lower=1> N_rad;                          // total radical-reform responses
  array[N_rad] int<lower=1, upper=N>    i_rad;  // respondent index
  array[N_rad] int<lower=1, upper=J_rad> j_rad; // item index
  array[N_rad] real                     y_rad; // observed (standardized)
}

parameters {
  // A) Latent-factor hierarchy (3-dimensional now: φ, θ, ψ)
  cholesky_factor_corr[3] Lcorr_eta;  // corr(φ, θ, ψ)
  vector<lower=0>[3]      tau_eta;    // half-Normal(0, 0.3)
  matrix[3, N]            z_eta;      // non-centered

  // 2) Region intercepts (3-dimensional)
  cholesky_factor_corr[3] Lcorr_alpha; // corr across (α₁, α₂, α₃)
  matrix[3, R]            z_alpha;     // non-centered
  vector<lower=0>[3]      sigma_alpha; // SD ≥ 0

  // 3) Party intercepts (3-dimensional)
  cholesky_factor_corr[3] Lcorr_delta; // corr across (δ₁, δ₂, δ₃)
  matrix[3, Q]            z_delta;     // non-centered
  vector<lower=0>[3]      sigma_delta; // SD ≥ 0

  // 4) Covariate slopes (3 × P)
  matrix[3, P]            B; // Normal(0, 0.5)

  // B) Measurement: optimism items
  vector[J_opt]           beta_opt;    // intercepts
  vector<lower=0>[J_opt]  lambda_opt;  // loadings ≥ 0
  vector<lower=0>[J_opt]  sigma_opt;   // residual SD ≥ 0

  // C) Measurement: environment items
  vector[J_env]           beta_env;    // intercepts
  vector<lower=0>[J_env]  lambda_env;  // loadings ≥ 0
  vector<lower=0>[J_env]  sigma_env;   // residual SD ≥ 0

  // D) Measurement: radical-reform items
  vector[J_rad]           beta_rad;    // intercepts
  vector<lower=0>[J_rad]  lambda_rad;  // loadings ≥ 0
  vector<lower=0>[J_rad]  sigma_rad;   // residual SD ≥ 0
}

transformed parameters {
  // Expand region & party covariance matrices
  cov_matrix[3] Sigma_alpha =
    diag_pre_multiply(sigma_alpha, Lcorr_alpha)
    * diag_pre_multiply(sigma_alpha, Lcorr_alpha)';
  cov_matrix[3] Sigma_delta =
    diag_pre_multiply(sigma_delta, Lcorr_delta)
    * diag_pre_multiply(sigma_delta, Lcorr_delta)';

  // Vectors for each latent dimension
  vector[N] phi;
  vector[N] theta;
  vector[N] psi;

  for (i in 1:N) {
    // region + party intercept contributions (3-vector)
    vector[3] mu_eta_i =
      diag_pre_multiply(sigma_alpha, Lcorr_alpha) * z_alpha[, region_id[i]] +
      diag_pre_multiply(sigma_delta, Lcorr_delta) * z_delta[, party_id[i]] +
      B * to_vector(X[i]);

    // latent noise (3-vector)
    vector[3] noise =
      diag_pre_multiply(tau_eta, Lcorr_eta) * z_eta[, i];

    vector[3] eta_i = mu_eta_i + noise;
    phi[i]   = eta_i[1];
    theta[i] = eta_i[2];
    psi[i]   = eta_i[3];
  }
}

model {
  // 1) Priors on τ_eta (latent SDs)
  // Optimism (φ) and Environment (θ): existing tuned priors
  tau_eta[1] ~ normal(0, 0.3);
  tau_eta[2] ~ normal(0, 0.3);
  
  // Radical Reform (ψ): tighter prior
  tau_eta[3] ~ normal(0, 0.1);  

  Lcorr_eta        ~ lkj_corr_cholesky(2.0);
  to_vector(z_eta) ~ normal(0, 1);

  // 2) Region intercept priors
  Lcorr_alpha      ~ lkj_corr_cholesky(2.0);
  sigma_alpha      ~ normal(0, 0.1) T[0, ];
  to_vector(z_alpha) ~ normal(0, 1);

  // 3) Party intercept priors
  Lcorr_delta      ~ lkj_corr_cholesky(2.0);
  sigma_delta      ~ normal(0, 0.1) T[0, ];
  to_vector(z_delta) ~ normal(0, 1);

  // 4) Covariate slopes
  to_vector(B)     ~ normal(0, 0.5);

  // 5) Measurement: optimism
  beta_opt        ~ normal(0, 0.5);
  lambda_opt      ~ lognormal(log(1), 0.2);
  sigma_opt       ~ normal(1, 0.2);

  // 6) Measurement: environment
  beta_env        ~ normal(0, 0.5);
  lambda_env      ~ lognormal(log(1), 0.2);
  sigma_env       ~ normal(1, 0.2);

  // 7) Measurement: radical-reform
  beta_rad        ~ normal(0, 0.5);
  lambda_rad      ~ lognormal(log(1), 0.2);
  sigma_rad       ~ normal(1, 0.2);

  // 8) Likelihood: y_opt
  for (n in 1:N_opt) {
    int ii = i_opt[n];
    int jj = j_opt[n];
    real mu_opt = beta_opt[jj] + lambda_opt[jj] * phi[ii];
    y_opt[n] ~ normal(mu_opt, sigma_opt[jj]);
  }

  // 9) Likelihood: y_env
  for (n in 1:N_env) {
    int ii = i_env[n];
    int jj = j_env[n];
    real mu_env = beta_env[jj] + lambda_env[jj] * theta[ii];
    y_env[n] ~ normal(mu_env, sigma_env[jj]);
  }

  // 10) Likelihood: y_rad
  for (n in 1:N_rad) {
    int ii = i_rad[n];
    int jj = j_rad[n];
    real mu_rad = beta_rad[jj] + lambda_rad[jj] * psi[ii];
    y_rad[n] ~ normal(mu_rad, sigma_rad[jj]);
  }
}

generated quantities {
  // A) Posterior-predictive y's for each block
  vector[N_opt] y_opt_sim;
  for (n in 1:N_opt) {
    int ii = i_opt[n];
    int jj = j_opt[n];
    real mu_opt = beta_opt[jj] + lambda_opt[jj] * phi[ii];
    y_opt_sim[n] = normal_rng(mu_opt, sigma_opt[jj]);
  }

  vector[N_env] y_env_sim;
  for (n in 1:N_env) {
    int ii = i_env[n];
    int jj = j_env[n];
    real mu_env = beta_env[jj] + lambda_env[jj] * theta[ii];
    y_env_sim[n] = normal_rng(mu_env, sigma_env[jj]);
  }

  vector[N_rad] y_rad_sim;
  for (n in 1:N_rad) {
    int ii = i_rad[n];
    int jj = j_rad[n];
    real mu_rad = beta_rad[jj] + lambda_rad[jj] * psi[ii];
    y_rad_sim[n] = normal_rng(mu_rad, sigma_rad[jj]);
  }

  // B) R² computations for each latent block
  // 1) R²_opt
  vector[N_opt] yhat_opt_resp;       
  vector[N_opt] var_resid_opt_resp;  

  for (n in 1:N_opt) {
    int ii = i_opt[n];
    int jj = j_opt[n];
    real mu_opt_n = beta_opt[jj] + lambda_opt[jj] * phi[ii];
    yhat_opt_resp[n]      = mu_opt_n;
    var_resid_opt_resp[n] = square(sigma_opt[jj]);
  }

  real Var_pred_opt = variance(yhat_opt_resp);
  real E_resid_opt  = mean(var_resid_opt_resp);
  real R2_opt       = Var_pred_opt / (Var_pred_opt + E_resid_opt);

  // 2) R²_env
  vector[N_env] yhat_env_resp;
  vector[N_env] var_resid_env_resp;

  for (n in 1:N_env) {
    int ii = i_env[n];
    int jj = j_env[n];
    real mu_env_n = beta_env[jj] + lambda_env[jj] * theta[ii];
    yhat_env_resp[n]      = mu_env_n;
    var_resid_env_resp[n] = square(sigma_env[jj]);
  }

  real Var_pred_env = variance(yhat_env_resp);
  real E_resid_env  = mean(var_resid_env_resp);
  real R2_env       = Var_pred_env / (Var_pred_env + E_resid_env);

  // 3) R²_rad
  vector[N_rad] yhat_rad_resp;
  vector[N_rad] var_resid_rad_resp;

  for (n in 1:N_rad) {
    int ii = i_rad[n];
    int jj = j_rad[n];
    real mu_rad_n = beta_rad[jj] + lambda_rad[jj] * psi[ii];
    yhat_rad_resp[n]      = mu_rad_n;
    var_resid_rad_resp[n] = square(sigma_rad[jj]);
  }

  real Var_pred_rad = variance(yhat_rad_resp);
  real E_resid_rad  = mean(var_resid_rad_resp);
  real R2_rad       = Var_pred_rad / (Var_pred_rad + E_resid_rad);
}
\end{verbatim}

\end{document}